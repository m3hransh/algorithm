\section{Brute Force}
Our first solution is to check all the possible 
subarrays. Every subarray is consists of a start 
and an end. So there are $\binom{n}{2} + n $
possible sub-arrays that is needed to consider
( n is for subarray of length 1). 
This can be implemented with two for-loops. The 
first for-loop chooses different possible 
values for the beginning of the sub-array 
and the second one values for the end of 
that sub-array. The implementation is in 
\textbf{Listing \ref{list:brute-force}}.
\pythonexternal[caption={\textbf{Brute force implemented of maximum subarray sum}},
    label={list:brute-force}]{codes/brute_force.py}
As you can see, for-loops go through all the possible 
subarrays, and for each, only it takes $\Theta(1)$ 
to check if it is the maximum or not. So in total 
takes $\Theta(n^2)$ to find the maximum subarray.
